\subsection{Translation Stage I: Higher-order Functional to Higher-order Imperative}
\label{sec:translation-i}

The use of assignment at compound data types is defined by:
\begin{displaymath}
  \begin{array}{lcl}
    A :=_{\num} E
    &=& A := E \\
    A :=_{N.\delta} E
    &=& \prim{mapParI}~N~\delta~\delta~(\lambda x~a.a :=_\delta x)~E~A \\
    A :=_{\delta_1 \times \delta_2} E
    &=& \prim{pairAcc_1}~A :=_{\delta_1} \prim{fst}~E;
        \prim{pairAcc_2}~A :=_{\delta_2} \prim{snd}~E
  \end{array}
\end{displaymath}

It is important that these translations satisfy the following equivalences.
{\small\begin{mathpar}
  \semA{E}_\delta(A) \simeq A :=_\delta E

  \semE{E}_\delta(C) \simeq C(E)
\end{mathpar}}

\begin{figure}[H]
  \begin{displaymath}
    \begin{array}{lcl}
      \semA{x}_\delta(A)
      &=& A :=_\delta x
      \medskip\\

      \semA{\underline{\ell}}_{\num}(A)
      &=& A := \underline{\ell}
      \\
      \semA{\prim{negate}~E}_{\num}(A)
      &=& \semE{E}_{\num}(\lambda x.~A := \prim{negate}~x)
      \\
      \semA{E_1 + E_2}_{\num}(A)
      &=& \semE{E_1}_{\num}(\lambda x.~\semE{E_2}_{\num}(\lambda y.~A := x + y))
      \medskip\\

      \semA{\prim{mapPar}~N~\delta_1~\delta_2~F~E}_{N.\delta_2}(A)
      &=& \semE{E}_{N.\delta_1}(\lambda x.~\prim{mapParI}~N~\delta_1~\delta_2~(\lambda x~o. \semA{F~x}_{\delta_2}(o))~x~A)
      \\
      \semA{\prim{mapSeq}~N~\delta_1~\delta_2~F~E}_{N.\delta_2}(A)
      &=& \semE{E}_{N.\delta_1}(\lambda x.~\prim{mapSeqI}~N~\delta_1~\delta_2~(\lambda x~o. \semA{F~x}_{\delta_2}(o))~x~A)
      \\
      \semA{\prim{reduceSeq}~N~\alpha~\delta_1~\delta_2~F~I~E}_{\mathsf{num}}(A)
      &=& \begin{array}[t]{@{}l}
            \semE{E}_{N.\delta_1}(\lambda x.~\semE{I}_{\delta_2}(\lambda y.\\
            \quad\quad \prim{reduceSeqI}~N~\alpha~\delta_1~\delta_2~(\lambda x~y~o. \semA{F~x~y}_{\delta_2}(o))~y~x~(\lambda r.A := r)))
          \end{array}
      \medskip\\

      \semA{\prim{split}~N~M~\delta~E}_{N.M.\delta}(A)
      &=& \semA{E}_{NM.\delta}(\prim{splitAcc}~N~M~\delta~A)
      \\
      \semA{\prim{join}~N~M~\delta~E}_{NM.\delta}(A)
      &=& \semA{E}_{N.M.\delta}(\prim{joinAcc}~N~M~\delta~A)
      \\
      \semA{\prim{zip}~N~\delta_1~\delta_2~E_1~E_2}_{N.\delta_1\times\delta_2}(A)
      &=& \semA{E_1}_{N.\delta_1}(\prim{zipAcc}_1~N~\delta_1~\delta_2~A);
          \semA{E_2}_{N.\delta_2}(\prim{zipAcc}_2~N~\delta_1~\delta_2~A)
      \\
      \semA{\prim{pair}~\delta_1~\delta_2~E_1~E_2}_{\delta_1\times\delta_2}(A)
      &=& \semA{E_1}_{\delta_1}(\prim{pairAcc_1}~\delta_1~\delta_2~A);
          \semA{E_2}_{\delta_2}(\prim{pairAcc_2}~\delta_1~\delta_2~A)
      \\
      \semA{\prim{fst}~{\delta_1}~{\delta_2}~E}_{\delta_1}(A)
      &=& \semE{E}_{\delta_1\times\delta_2}(\lambda x.~A :=_{\delta_1} \prim{fst}~{\delta_1}~{\delta_2}~x)
      \\
      \semA{\prim{snd}~{\delta_1}~{\delta_2}~E}_{\delta_2}(A)
      &=& \semE{E}_{\delta_1\times\delta_2}(\lambda x.~A :=_{\delta_2} \prim{snd}~{\delta_1}~{\delta_2}~x)
    \end{array}
  \end{displaymath}
  \caption{Acceptor-passing Translation}\label{fig:acc-trans}
\end{figure}

\begin{figure}[H]
  \begin{displaymath}
    \begin{array}{lcl}
      \semE{x}_\delta(C)
      &=& C(x)
      \medskip\\
      \semE{\underline{\ell}}_{\num}(C)
      &=& C(\underline{\ell})
      \\
      \semE{\prim{negate}~E}_{\num}(C)
      &=& \semE{E}_{\num}(\lambda x.~C(\prim{negate}~x))
      \\
      \semE{E_1 + E_2}_{\num}(C)
      &=& \semE{E_1}_{\num}(\lambda x.~\semE{E_2}_{\num}(\lambda y.~C(x+y)~)~)
      \medskip\\

      \semE{\prim{toMem}~\alpha~\delta~E}_{\delta}(C)
        &=& \prim{new}~\alpha~\delta~(\lambda \mathit{tmp}.~\semA{E}_{\delta}(\mathit{tmp}.w);~C(\mathit{tmp}.r)~)
      \\
      \semE{\prim{reduceSeq}~N~\alpha~\delta_1~\delta_2~F~I~E}_{\delta_2}(C)
      &=& \begin{array}[t]{@{}l}
            \semE{E}_{N.\delta_1}(\lambda x.~
            \semE{I}_{\delta_2}(\lambda y.\\
      \quad\quad \prim{reduceSeqI}~N~\alpha~\delta_1~\delta_2~(\lambda x~y~o.~\semA{F~x~y}_{\delta_2}(o))~y~x)~C)
          \end{array}
      \medskip\\

      \semE{\prim{split}~N~M~\delta~E}_{N.M.\delta}(C)
      &=& \semE{E}_{NM.\delta}(\lambda x.~C(\prim{split}~N~M~\delta~x)~)
      \\
      \semE{\prim{join}~N~M~\delta~E}_{NM.\delta}(C)
      &=& \semE{E}_{N.M.\delta}(\lambda x.~C(\prim{join}~N~M~\delta~x)~)
      \\
      \semE{\prim{zip}~N~\delta_1~\delta_2~E_1~E_2}_{N.\delta_1 \times \delta_2}(C)
      &=& \semE{E_1}_{N.\delta_1}(\lambda x.~\semE{E_2}_{N.\delta_2}(\lambda y.~C(\prim{zip}~N~\delta_1~\delta_2~x~y)~)~)
      \\
      \semE{\prim{pair}~\delta_1~\delta_2~E_1~E_2}_{\delta_1\times\delta_2}(C)
      &=& \semE{E_1}_{\delta_1}(\lambda x.~\semE{E_2}_{\delta_2}(\lambda y.~C(\prim{pair}~\delta_1~\delta_2~x~y)~)~)
      \\
      \semE{\prim{fst}~\delta_1~\delta_2~E}_{\delta_1}(C)
      &=& \semE{E}_{\delta_1\times\delta_2}(\lambda x.~C(\prim{fst}~\delta_1~\delta_2~x)~)
      \\
      \semE{\prim{snd}~\delta_1~\delta_2~E}_{\delta_2}(C)
      &=& \semE{E}_{\delta_1\times\delta_2}(\lambda x.~C(\prim{snd}~\delta_1~\delta_2~x)~)
      \\
    \end{array}
  \end{displaymath}
  \subcaption{Continuation-passing Translation}\label{fig:con-trans}
\end{figure}

\subsubsection{Map fusion} % TODO: discuss the terminology

We now describe how we translate \prim{map} primitives that should be fused into other primitives.

For the acceptor translation, the intuition is that we cannot decide what to do by solely deconstructing the \prim{map} primitive. The \prim{map} should be fused into its input and we have no generic way of solving this, which is why the result of the translation also depends on the input. We introduce an acceptor-passing map translation where we deconstruct the input:
\begin{align*}
\semA{\prim{map}~N~\delta_1~\delta_2~F~E}_{N.\delta_2}(A)
&= \semM{F}{E}_{N.\delta_2}(A) \\
&\simeq A :=_{N.\delta_2} \prim{map}~N~\delta_1~\delta_2~F~E
\end{align*}

\begin{figure}[H]
  \begin{displaymath}
    \begin{array}{lcl}
      \semM{G}{\prim{map}~N~\delta_1~\delta_2~F~E}_{N.\delta_3}(A)
      &=& \semM{\lambda x.~G~(F~x)}{E}_{N.\delta_3}(A) % TODO? composition symbol
      \\
      \semM{G}{\prim{mapPar}~N~\delta_1~\delta_2~F~E}_{N.\delta_3}(A)
      &=& \semE{E}_{N.\delta_1}(\lambda x.~\prim{mapParI}~N~\delta_1~\delta_2~(\lambda x~o. \semA{G~(F~x)}_{\delta_3}(o))~x~A)
      \\
      \semM{G}{\prim{mapSeq}~N~\delta_1~\delta_2~F~E}_{N.\delta_3}(A)
      &=& \semE{E}_{N.\delta_1}(\lambda x.~\prim{mapSeqI}~N~\delta_1~\delta_2~(\lambda x~o. \semA{G~(F~x)}_{\delta_3}(o))~x~A)
      \medskip\\

      \semM{G}{\prim{join}~N~M~\delta_1~E}_{NM.\delta_2}(A)
      &=& \semM{\prim{map}~M~\delta_1~\delta_2~G}{E}_{N.M.\delta_2}(\prim{joinAcc}~N~M~\delta_1~A)
      \\
    \end{array}
  \end{displaymath}
  \caption{Acceptor-passing map translation}\label{fig:map-acc-trans}
\end{figure}

For the continuation translation, the intuition is that, given an element input, we can continue some later translation with the output of the function. We introduce the \prim{mapRead} primitive which is passed through to the following stages:
$$\semE{\prim{map}~N~\delta_1~\delta_2~F~E}_{N.\delta_2}(C) = \semE{E}_{N.\delta_1}(\lambda x.~C(\prim{mapRead}~N~\delta_1~\delta_2~(\lambda a~cont.~\semE{F~a}_{\delta_2}(cont))~x)) $$
$$ \prim{mapRead}: (n : \natkind) \to (\dtone~\dttwo : \datatype) \to (\tyexp[\dtone] \to (\tyexp[\dttwo] \to \tycomm) \to \tycomm) \to \tyexp[n.\dtone] \to \tyexp[n.\dttwo] $$

\clearpage

\subsection{Translation Stage II: Higher-order Imperative to For-loops}
\label{sec:translation-ii}

\newcommand{\ei}{\prim{idx}~n~\dtone~\mathit{xs}~i}
\newcommand{\ea}{\prim{idxAcc}~n~\dttwo~\mathit{out}~i}

\begin{displaymath}
  \prim{mapParI}
  =
  \Lambda n~\dtone~\dttwo.\lambda f\,\mathit{xs}~\mathit{out}.~
  \prim{parfor}~n~\dttwo~\mathit{out}~(\lambda i~o.~f\,(\ei)~o)
\end{displaymath}
\begin{displaymath}
  \prim{mapSeqI}
  =
  \Lambda n~\dtone~\dttwo.\lambda f\,\mathit{xs}~\mathit{out}.~
  \prim{for}~n~(\lambda~i.~f\,(\ei)~(\ea)\ )
\end{displaymath}
\begin{displaymath}
  \prim{reduceSeqI}
  =
  \Lambda n~a~\dtone~\dttwo.\lambda f\,\mathit{init}~\mathit{xs}~c.~\prim{new}~a~\dttwo~(\lambda \mathit{acc}.~
    \begin{array}[t]{@{}l}
    \semA{init}(acc.w)\\
    \prim{for}~n~(\lambda i.~f\,(\ei)~(acc.r)~(acc.w)\ );\\
    c~(acc.r)\ )
    \end{array}
\end{displaymath}

\clearpage

\subsection{Translation Stage III: For-loops to Parallel Pseudo-C}
\label{sec:translation-iii}

% TODO: updated codegen

\newcommand{\codegenComm}[1]{\textsc{CodeGen}_{\mathsf{comm}}(#1)}
\newcommand{\codegenAcc}[2]{\textsc{CodeGen}_{\mathsf{acc}[#1]}(#2)}
\newcommand{\codegenExp}[2]{\textsc{CodeGen}_{\mathsf{exp}[#1]}(#2)}
\newcommand{\codegenData}[1]{\textsc{CodeGen}_{\mathsf{\datatype}}(#1)}

\begin{figure}[H]
  \begin{minipage}{1.0\linewidth}
    \begin{displaymath}
      \begin{array}{l@{\hspace{0.4em}}c@{\hspace{0.4em}}l}
        \codegenComm{\prim{skip}, \eta}
        &=& \texttt{/* skip */}
        \\
        \codegenComm{P_1; P_2, \eta}
        &=& \begin{array}[t]{@{}l}
              \codegenComm{P_1, \eta}~\codegenComm{P_2, \eta}
            \end{array}
        \\
        \codegenComm{A := E, \eta}
        &=& \codegenAcc{\num}{A, \eta, []}\texttt{ = }\codegenExp{\num}{E, \eta, []}\texttt{;}
        \\
        \codegenComm{\prim{new}~\delta~(\lambda v.~P), \eta}
        &=& \texttt{\{}~
            \begin{array}[t]{@{}l}
              \codegenData{\delta}~\texttt{v;} \\
              \codegenComm{P[(v_a,v_e)/v], \eta[v_a \mapsto \texttt{v}, v_e \mapsto \texttt{v}]}\ \texttt{\}}
            \end{array}
        \\
        \codegenComm{\prim{for}~n~(\lambda i.~P), \eta}
        &=& \begin{array}[t]{@{}l}
              \texttt{for(int i = 0; i < $n$; i += 1) \{} \\
              \quad \codegenComm{P, \eta[i \mapsto \texttt{i}]} \ \texttt{\}}
            \end{array}
        \\
        \codegenComm{\prim{parfor}~n~\delta~A~(\lambda i~o.~P)~E, \eta}
        &=& \begin{array}[t]{@{}l}
              \texttt{\#pragma omp parallel for}\\
              \texttt{for(int i = 0; i < $n$; i += 1) \{ }\\
              \quad \codegenComm{P[\prim{idxAcc}~n~\delta~A~i/o], \eta[i \mapsto \texttt{i}]}\ \texttt{\}}
            \end{array}
      \end{array}
    \end{displaymath}
    \subcaption{\DPIA commands to OpenMP C statements}\label{fig:codegen-comm}
  \end{minipage}

  \bigskip

  \begin{minipage}{1.0\linewidth}
    \begin{displaymath}
      \begin{array}{@{}l@{~}c@{~}l@{}}
        \codegenAcc{\delta}{x, \eta, \mathit{ps}}
        &=& \eta(x)(\mathit{reverse}~\mathit{ps})
        \\
        \codegenAcc{\mathsf{\delta}}{\prim{idxAcc}~N~\delta~A~I, \eta, \mathit{ps}}
        &=& \begin{array}[t]{@{}l}
          \codegenAcc{N.\delta}{A, \eta,\\ \quad\codegenExp{\tyidx[n]}{I, \eta, []} :: \mathit{ps})}
        \end{array}
        \\
        \codegenAcc{NM.\delta}{\prim{splitAcc}~N~M~\delta~A, \eta, \texttt{i} :: \mathit{ps}}
        &=& \codegenAcc{N.M.\delta}{A, \eta, \texttt{i/}M :: \texttt{i\%}M :: \mathit{ps}}
        \\
        \codegenAcc{N.M.\delta}{\prim{joinAcc}~n~M~\delta~A, \eta, \texttt{i} :: \texttt{j} :: \mathit{ps}}
        &=& \codegenAcc{NM.\delta}{A, \eta, \texttt{i*}M\texttt{+j} :: \mathit{ps}}
        \\
        \codegenAcc{\delta_1}{\prim{pairAcc_1}~\delta_1~\delta_2~A, \eta, \mathit{ps}}
        &=& \codegenAcc{\delta_1\times\delta_2}{A, \eta, \texttt{.x1} :: \mathit{ps}}
        \\
        \codegenAcc{\delta_2}{\prim{pairAcc_2}~\delta_1~\delta_2~A, \eta, \mathit{ps}}
        &=& \codegenAcc{\delta_1\times\delta_2}{A, \eta, \texttt{.x2} :: \mathit{ps}}
        \\
        \codegenAcc{N.\delta_1}{\prim{zipAcc_1}~N~\delta_1~\delta_2~A, \eta, \texttt{i} :: \mathit{ps}}
        &=& \codegenAcc{N.(\delta_1\times\delta_2)}{A, \eta, \texttt{i} :: \texttt{.x1} :: \mathit{ps}}
        \\
        \codegenAcc{N.\delta_2}{\prim{zipAcc_2}~N~\delta_1~\delta_2~A, \eta, \texttt{i} :: \mathit{ps}}
        &=& \codegenAcc{N.(\delta_1\times\delta_2)}{A, \eta, \texttt{i} :: \texttt{.x2} :: \mathit{ps}}
      \end{array}
    \end{displaymath}
    \subcaption{\DPIA acceptors to C l-values}\label{fig:codegen-acc}
  \end{minipage}

  \bigskip

  \begin{minipage}{1.0\linewidth}
    \begin{displaymath}
      \begin{array}{@{}l@{~}c@{~}l@{}}
        \codegenExp{\delta}{x, \eta, \mathit{ps}}
        &=& \eta(x)(\mathit{reverse}~\mathit{ps})
        \\
        \codegenExp{\num}{\underline{\ell}, \eta, []}
        &=& \ell
        \\
        \codegenExp{\num}{\prim{negate}~E, \eta, []}
        &=& \texttt{(}\texttt{-}~\codegenExp{\num}{E, \eta, []}\texttt{)}
        \\
        \codegenExp{\num}{E_1 + E_2, \eta, []}
        &=& \texttt{(}
        \begin{array}[t]{@{}l}
          \codegenExp{\num}{E_1, \eta, []}\\
          \texttt{+ }\codegenExp{\num}{E_2, \eta, []}\texttt{)}
        \end{array}
        \\
        \codegenExp{N.(\delta_1 \times \delta_2)}{\prim{zip}~N~\delta_1~\delta_2~E_1~E_2, \eta, \texttt{i} :: \texttt{.x}j :: \mathit{ps}}
        &=& \codegenExp{N.\delta_j}{E_j, \eta, \texttt{i} :: \mathit{ps}}
        \\
        \codegenExp{M.N.\delta}{\prim{split}~N~M~\delta~E, \eta, \texttt{i} :: \texttt{j} :: \mathit{ps}}
        &=& \codegenExp{MN.\delta}{E, \eta, \texttt{i*}n\texttt{+j} :: \mathit{ps}}
        \\
        \codegenExp{MN.\delta}{\prim{join}~N~M~\delta~E, \eta, \texttt{i} :: \mathit{ps}}
        &=& \codegenExp{M.N.\delta}{E, \eta, \texttt{i/}n :: \texttt{i\%}n :: \mathit{ps}}
        \\
        \codegenExp{\delta_1 \times \delta_2}{\prim{pair}~\delta_1~\delta_2~E_1~E_2, \eta, \texttt{.x}j :: \mathit{ps}}
        &=& \codegenExp{\delta_j}{E_j, \eta, \mathit{ps}}
        \\
        \codegenExp{\delta_1}{\prim{fst}~\delta_1~\delta_2~E, \eta, \mathit{ps}}
        &=& \codegenExp{\delta_1 \times \delta_2}{E, \eta, \texttt{.x1} :: \mathit{ps}}
        \\
        \codegenExp{\delta_2}{\prim{snd}~\delta_1~\delta_2~E, \eta, \mathit{ps}}
        &=& \codegenExp{\delta_1 \times \delta_2}{E, \eta, \texttt{.x2} :: \mathit{ps}}
        \\
        \codegenExp{\delta}{\prim{idx}~N~\delta~E~I, \eta, \mathit{ps}}
        &=&
        \begin{array}[t]{@{}l}
          \textsc{CodeGen}_{\mathsf{exp}[N.\delta]}(E, \eta,\\ \quad\codegenExp{\tyidx[N]}{I, \eta, []} :: \mathit{ps})
        \end{array}
      \end{array}
    \end{displaymath}
    \subcaption{\DPIA expressions to C r-values}\label{fig:codegen-exp}
  \end{minipage}

  \caption{Translation of Purely Imperative \DPIA to parallel OpenMP C}
  \label{fig:codegen}
\end{figure}
